\documentclass[11pt]{article}

\usepackage{url}
\usepackage{multicol}
\usepackage[english]{babel}
\usepackage[margin=1in]{geometry}
\usepackage{graphicx}
\usepackage{subcaption}
\usepackage{enumitem}
\usepackage{amsmath}
\usepackage{amssymb}
\usepackage{wasysym}
\usepackage{color}
\usepackage{float}
\usepackage{nomencl}
\usepackage[title]{appendix}
\makenomenclature
\usepackage{pdfpages}
\usepackage{algorithm}
\usepackage{algpseudocode}
\usepackage{hyperref}
\hypersetup{
    colorlinks=true,
    linkcolor=blue,
    filecolor=magenta,      
    urlcolor=cyan,
    pdftitle={Overleaf Example},
    pdfpagemode=FullScreen,
    }
\title{16-745 Optimal Control Lecture 8}
\author{Reid Graves} 

\begin{document}
\maketitle

\section{Last Time}
\begin{itemize}
    \item Deterministic Optimal Control
    \item Pontryagin
    \item Indirect Shooting
\end{itemize}

\section{Today}
\begin{itemize}
    \item LQR Problem
    \item LQR as a QP
    \item Riccati Recursion
\end{itemize}

\section{LQR Problem}
\begin{align*}
    \min_{x_{1:N},u_{1:N-1}} \quad J &= \sum_{k=1}^{N-1}\frac{1}{2}x_k^TQ_kx_k + \frac{1}{2}u_k^TRu_k + \frac{1}{2}x_N^TQ_Nx_N
    \\
    s.t. \quad x_{k+1} &= A_kx_k + B_ku_k
    \\
    Q_k &\geq 0, \quad R>0
\end{align*}
$R$ needs to be greater than 0 for problem to be well posed. 

\subsection{Example}
\begin{itemize}
    \item ``Double Integrator"
    \begin{align*}
        \dot{x} &= \begin{bmatrix}
            \dot{q} \\
            \ddot{q}
        \end{bmatrix}
        =
        \begin{bmatrix}
            0 & 1 \\
            0 & 0 
        \end{bmatrix}
        \begin{bmatrix}
            q \\
            \dot{q}
        \end{bmatrix}
        + \begin{bmatrix}
            0 \\
            1
        \end{bmatrix}
        u
    \end{align*}
    \item Think of this as a brick sliding on ice (no friction)
    \begin{align*}
        x_{k+1} &= 
        \begin{bmatrix}
            1 & h \\
            0 & 1
        \end{bmatrix}
        \begin{bmatrix}
            q_k \\
            \dot{q}_k
        \end{bmatrix}
        +
        \begin{bmatrix}
            \frac{1}{2}h^2 \\
            h
        \end{bmatrix}
        u_k
        \\
        x_{k+1} &= Ax_k + Bu_k
    \end{align*}
    Where $h$ is the timestep
\end{itemize}

\section{LQR as a QP}
\begin{itemize}
    \item Assume $x_1$ (initial state) is given (not a decision variable)
    \item Define 
    \begin{align*}
        z &= 
        \begin{bmatrix}
            u_1 \\
            x_2 \\
            u_2 \\
            x_3 \\
            \vdots \\
            x_N
        \end{bmatrix}
    \end{align*}
    \begin{align*}
        H &= 
        \begin{bmatrix}
            R_1  & & & & \\
            & Q_2 & & 0&  \\
            & & R_2 & & \\
            & 0& & \ddots & \\
            & & & & Q_N
        \end{bmatrix}
    \end{align*}
    such that $J = \frac{1}{2}Z^THZ$
    \item Define $C$ and $d$:
    \begin{align*}
        &\begin{bmatrix}
            B_1 & -I & 0 & \dots & \dots & & 0 \\
            0 & A_2 & B_2 & -I & 0 & \dots & 0 \\  
            & & & \ddots & & & \\
            & & & & A_{N-1} & B_{N-1} & -I
        \end{bmatrix}
        \begin{bmatrix}
            u_1 \\
            x_2 \\
            u_2 \\
            \vdots \\
            x_N
        \end{bmatrix}
        =
        \begin{bmatrix}
            -Ax_1 \\
            0 \\
            \vdots \\
            0
        \end{bmatrix}
        \\
        &\Rightarrow Cz = d
    \end{align*}
    \item Now we can write LQR as a standard QP:
    \begin{align*}
        \min_z \quad \frac{1}{2}z^THz
        \\
        s.t. \quad Cz &= d
    \end{align*}
    \item The Lagrangian of this QP is:
    \begin{align*}
        L(z,\lambda) &= \frac{1}{2}z^THz + \lambda^T[Cz-d]
    \end{align*}
    \item  KKT conditions:
    \begin{align*}
        \nabla_zL &= Hz + C^T\lambda = 0
        \\
        \nabla_\lambda L &= Cz-d = 0
        \\
        \Rightarrow \begin{bmatrix}
            H & C^T \\
            C & 0
        \end{bmatrix}
        \begin{bmatrix}
            z \\
            \lambda
        \end{bmatrix}
        &=
        \begin{bmatrix}
            0 \\
            d
        \end{bmatrix}
    \end{align*}
    \item We get the exact solution by solving one linear system!
\end{itemize}
\subsection{Example}
\begin{itemize}
    \item Much better than shooting!
\end{itemize}

\subsection{A closer look at the LQR QP}
\begin{itemize}
    \item The KKT system for LQR is very sparse (lots of zeros in matrix) and has lots of structure:
    \begin{align*}
    &\begin{bmatrix}
        R & & & & & B^T & & & &  \\
        & Q & & & & I & A^T & &  \\
        & & R & & & & & B^T & \\
        & & & Q & & & & -I& A^T\\
        & & & & R & & & & B^T\\
        & & & & & Q_N & & & -I\\
        B & -I & & & & & & \\
        & A & B & -I& & & & & 0 \\
        & & & A & B & -I & & &
    \end{bmatrix}
         \begin{bmatrix}
            u_1 \\
            x_2 \\
            u_2 \\
            x_3 \\
            u_3 \\
            x_4 \\
            \lambda_2 \\
            \lambda_3 \\
            \lambda_4
        \end{bmatrix}
    =
    \begin{bmatrix}
        \\
        0
        \\
        \\
        \\
        \\
        -Ax \\
        0 \\
        0
    \end{bmatrix}
    \\
   & Q_Nx_4 - \lambda_4 = 0 \Rightarrow \lambda_4 = Q_Nx_4
    \\
   & Ru_3 + B^T\lambda_4 = Ru_3 + B^TQ_Nx_4 = 0 \text{ Plug in dynamics for }x_4
    \\
   & \Rightarrow Ru_3 + B^TQ_N(Ax_3 + Bu_3) = 0
    \\
   & \Rightarrow u_3 = -(R + B^TQ_NB)^{-1}B^TQ_NAx_3 \text{ call big matrix before }x_3 \text{ }K_3 \\
   & Qx_3 - \lambda_3 + A^T\lambda^4 = 0 \text{ Plug in }\lambda_4
    \\
  &  Qx_3 - \lambda_3 + A^TQ_Nx_4 = 0 \text{ Plug in dynamics}
    \\
   & Qx_3 - \lambda_3 + A^TQ_N(Ax_3 + Bu_3) \text{ plug in } u_3=-K_3x_3
    \\
   & \Rightarrow \lambda_3 = (Q+A^TQ_N(A-BK_3))x_3 \text{ Call big matrix before }x_3 \text{ } P_3
    \end{align*}
    \item Now we have a recursion for $K$ and $P$:
    \begin{align*}
        P_N &= Q_N \\
        K_k &= (R+B^TP_{k+1}B)^{-1}B^TP_{n+1}A
        \\
        P_k &= Q + A^TP_{n+1}(A-BK_k)
    \end{align*}
    \item This is called the Riccati equation/recursion
    \item We can solve the QP by doing a backward Riccati pass followed by a forward rollout to compute $x_{1:N}$ and $u_{1:N-1}$
    \item General (dense) QP has complexity $O([N(n+m)]^3)$, Horizon $N$, state dimension $n$, control dim $m$
    \item Riccati solution is $O(N(n+m)^3)$
\end{itemize}
\subsection{Example}
\begin{itemize}
    \item Riccati exactly matches QP 
    \item Feedback policy lets us change $x_0$ and reject noise/disturbances
\end{itemize}

\subsection{Infinite Horizon LQR}
\begin{itemize}
    \item For time invariant LQR converge to constants
    \item For stabilization problems we usually use constant $K$
    \item Backward recursion for $P$:
    \begin{align*}
        k_k &= (R+B^TP_{n+1}B)^{-1}B^TP_{n+1}A
        \\
        P_k &= Q + A^TP_{k+1} (A+BK_k)
    \end{align*}
    \item Infinite horizon limit $\Rightarrow P_{k+1} = P_k = R_\text{inf}$ $\Rightarrow$
\end{itemize}



\end{document}
